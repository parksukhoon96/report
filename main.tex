\documentclass{article}
\usepackage[utf8]{inputenc}
\usepackage{kotex}
\usepackage{amsmath}
\usepackage{amssymb}
\usepackage{amsfonts}
\usepackage{graphicx}
\graphicspath{ {./images/} }
\usepackage{dcolumn}
\usepackage{bm}
\usepackage{hyperref}
\usepackage{mathptmx}
\usepackage{tikz}
\usetikzlibrary{positioning}
\usepackage{textcomp}
\usepackage{natbib}
\usepackage[rightcaption]{sidecap}
\usepackage{wrapfig}


\title{백악관}
\author{박석훈 / 학생 / 사회학과 ­}
\date{May 2022}

\begin{document}

\maketitle

\section{Introduction}

미국이 중국의 반도체 야심이 담긴 중국제조 2025를 저지하려는 움직임

Final Report National Security Commission on Artificial Intelligence


29. AI는 다음과 같은 세 가지 방법으로 미국 사회를 혼란으로 빠트릴 수 있다. 1) 메시지 2) 대중 3) 매체

악당과의 대결에서 같은 방식을 취할 것인가? 그러한 대립에 맞서는 또 다른 방안을 고수할 것인가?

그 방안을 고수한다면, 악당의 공격 패턴을 막을 수 있는 방법은? 

패권의 승자는 누구인가? 그러면 우리나라는/

미국이 가지고 있는 플랫폼

다양성 / 종의 기원 // 서로 다름을 인정하고 자유를 하는 것이 결국에는 승리하는 걱ㄴ가? 문제는 그만큼의 시간을 벌어다줄 수 있는가

디지털 전체주의
민주주의 세계에서 유권자의 표는 현금 이상의 가치가 있따.
\end{document}
