\documentclass[11pt, a4]{article}
\usepackage[utf8]{inputenc}
\usepackage{fancyhdr}
\usepackage{graphicx}
\PassOptionsToPackage{hyphens}{url}\usepackage{hyperref}
\usepackage{geometry}
\usepackage[
backend=biber,
style=ieee,
]{biblatex}
\usepackage{kotex}

\addbibresource{reference.bib}

\pagestyle{fancy}
\fancyhf{}
\setlength{\headheight}{14pt}
\rhead{Hestia}
\lhead{AI Report Project}
\cfoot{\thepage}

\begin{document}

\title{AI Report}
\author{sukhoon park}
\maketitle

\tableofcontents

\newpage
\section{Introduction}

작성(어제 메모 참고) 제언 다시 읽기

\subsection{Summary of the Report}

본 보고서는 4가지 영역에서 현 미국의 Artificial Intelligence (AI) 거버넌스\footnote{Governance refers, therefore, to all processes of governing, whether undertaken by a government, market, or network, whether over a family, tribe, formal or informal organization, or territory, and whether through laws, norms, power or language. Governance differs from government in that it focuses less on the state and its institutions and more on social practices and activities.\cite{bevir2012governance}}의 현황과 문제점, 향후 미래 전략을 4가지 영역에서 제시한다. 특히 보고서는 AI 패권경쟁의 주요 경쟁자인 중국과의 직접적인 비교를 통해 내용을 전개한다.

\subsubsection{백악관 리더십}

\subsubsection{인재확보 경쟁}

\subsubsection{하드웨어 경쟁}

\subsubsection{혁신적 투자}

\section{A Strategy for Competition and Cooperation}

\section{The Talent Competition}

\section{Accelerating AI Innovation}

\section{Intellectual Property}

\section{Microelectronics}

\section{Technology Protection}

\section{A Favorable International Technology Order}

\section{Associated Technologies}

\section{Conclusion}

\appendix

\newpage
\printbibliography

\end{document}